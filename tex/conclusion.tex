\documentclass[cacheSimReport.tex]{subfiles}
\begin{document}
\section*{\textsc{\Large Conclusion}}

The original goal of making and testing a cache simulator was completed fully. The implementation is clear and easy to follow, matches existing results up to 1 million traces, as well as provides meaningful and clear results. The simulations of the traces bzip2, h264ref, libquantum, omnetpp, and sjeng with the 11 different configurations provided extensive data to analyze the situation. Based on the concluding two graphs of the results section. The most efficient configuration is L2-Big. It balances execution cycles and cost better than the remainder of the configurations.

Simulating a cache structure was a valuable experiment in gaining knowledge about how caches work as well as learning how their performance can change in different situations. Working in a language like C gave insights into how memory needs to be managed to guarantee correct execution when some configurations can take an hour or more to run. Appended at the end of this report are all of the results as well as all of the code used to write the simulator.


\end{document}